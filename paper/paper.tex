%%%%%%%%%%%%%%%%%%%%%%%%%%%%%%%%%%%%%%%%%
%  My documentation report
%  Objetive: Explain what I did and how, so someone can continue with the investigation
%
% Important note:
% Chapter heading images should have a 2:1 width:height ratio,
% e.g. 920px width and 460px height.
%
%%%%%%%%%%%%%%%%%%%%%%%%%%%%%%%%%%%%%%%%%

%----------------------------------------------------------------------------------------
%	PACKAGES AND OTHER DOCUMENT CONFIGURATIONS
%----------------------------------------------------------------------------------------

\documentclass[11pt,fleqn]{book} % Default font size and left-justified equations

\usepackage[top=3cm,bottom=3cm,left=3.2cm,right=3.2cm,headsep=10pt,letterpaper]{geometry} % Page margins

\usepackage{xcolor} % Required for specifying colors by name
\definecolor{ocre}{RGB}{52,177,201} % Define the orange color used for highlighting throughout the book

% Font Settings
\usepackage{avant} % Use the Avantgarde font for headings
%\usepackage{times} % Use the Times font for headings
\usepackage{mathptmx} % Use the Adobe Times Roman as the default text font together with math symbols from the Sym­bol, Chancery and Com­puter Modern fonts

\usepackage{microtype} % Slightly tweak font spacing for aesthetics
\usepackage[utf8]{inputenc} % Required for including letters with accents
\usepackage[T1]{fontenc} % Use 8-bit encoding that has 256 glyphs

% Bibliography
\usepackage[style=alphabetic,sorting=nyt,sortcites=true,autopunct=true,babel=hyphen,hyperref=true,abbreviate=false,backref=true,backend=biber]{biblatex}
\addbibresource{bibliography.bib} % BibTeX bibliography file
\defbibheading{bibempty}{}

\input{structure} % Insert the commands.tex file which contains the majority of the structure behind the template

\begin{document}

%----------------------------------------------------------------------------------------
%	TITLE PAGE
%----------------------------------------------------------------------------------------

\begingroup
\thispagestyle{empty}
\AddToShipoutPicture*{\put(0,0){\includegraphics[scale=1.25]{frontpage3}}} % Image background
\centering
\vspace*{5cm}
\par\normalfont\fontsize{32}{32}\sffamily\selectfont
\textbf{Filmchat}\\
{\LARGE Movie recommendation as NUI}\par % Book title
\vspace*{1cm}
{\Huge Tarjei Utnes}\par % Author name
\endgroup

%----------------------------------------------------------------------------------------
%	TABLE OF CONTENTS
%----------------------------------------------------------------------------------------

\chapterimage{head3.png} % Table of contents heading image

\pagestyle{empty} % No headers

\tableofcontents % Print the table of contents itself

%\cleardoublepage % Forces the first chapter to start on an odd page so it's on the right

\pagestyle{fancy} % Print headers again

%----------------------------------------------------------------------------------------
%	CHAPTER 1 Project Description
%----------------------------------------------------------------------------------------

\chapterimage{head3.png} % Chapter heading image

\chapter{Project Description}
I am starting with the vision for this project. Imagine the following scenario.
You, and one or more friends, or maybe a romantic partner, are sitting int the sofa. You are having a movie night, but you are not quite sure what you should see. While you sit and discuss your likes, and dislikes, there is a microphone hidden in the corner. The listens to what you are saying, and it is sending it to an Interactive System. It understands, who is speaking, who is liking a movie, and who is not liking a movie. It also has, from earlier experience knowledge of what you and maybe the ones you are with have seen, and liked. It also knows what you did not like.

After a few minutes discussing with the people it starts rolling some suggestions over the television. 

By using your voice, you can tell the Interactive system to play a trailer, if it is showing you something you think might be interesting.

When you have decided it knows on what streaming, or maybe even traditional linear TV the movie plays on and selects that option for you.

The name of the system, is Filmchat.


This project is aimed for use at home, or possibly on-line in chat rooms or forums. The goal of the project is to interpret and come up with suggestions for which movies, or TV series its users should watch. I have limited the scope in this project to implement parts of the project, because of the available time given.
There are competing or projects existing that are close to doing what this project is aiming to do. Below is a non-exhaustive list of such projects, and solutions.
\section{“And Chill” Netflix recommendation bot}\index{AndChill}
The “And Chill” chatbot:
\begin{quote}
Well, that’s when the Facebook chatbot And Chill comes in handy. The bot asks you to suggest a film you like with a few reasons why, or to simply explain what you’re looking for in your next movie.  
Within a few moments, the bot should issue you with a short selection of trailers. It sometimes takes a while to compute, but it’s still a pretty efficient way of narrowing down Netflix’s massive library.
That said, if it’s running at peak capacity, you’ll have to wait your turn before it takes your requests. A representative for the app told a Facebook commenter last month that they were working on scaling capacity. (Huffington Post 2016)
\end{quote}
\section{Filmgrail}\index{Filmgrail}
Filmgrail is an app to check where you are able to watch the movie or television series you have selected. What makes Filmgrail unique in a Norwegian context, is that it covers many of the Norwegian streaming services. (Aftenposten, Filmgrail.com). At the moment, however 
\section{Google Assistant}\index{googleassistant}
Google Assistant can also recommend movies.



%----------------------------------------------------------------------------------------
%	CHAPTER 2 Interactive System Description
%----------------------------------------------------------------------------------------
\chapterimage{head3.png}

\chapter{Interactive System: Description}
There are two parts implemented in this system
The first part listens to a given channel on Slack, using the outgoing webhook (REFERENCE). 
The second part uses Googles Speech API (reference) and listens to the microphone. It then tries to search for the terms in the recognized speech and if it finds a movie within it it posts it to the same Slack channel.

The though of using both of these ways of integrating, was to be able to understand natural language surrounding how people talk and chat about movies, and then later on have the different parts use the same components.



%----------------------------------------------------------------------------------------
%	CHAPTER 3 Interactive System Design
%----------------------------------------------------------------------------------------

\chapterimage{head3.png}
\chapter{Interactive System: Design}
•	Sketches
•	Design rationale for the interface
•	Cost/benefit discussion of selected interaction techniques

Cost / Benefit of selected interaction techniques
Chat:
Chat is in my view has become a natural way of interacting, several times a day, I interact through chat, be it privately and also work related. A non-exhaustive list of some of the chat interaction I use during a day, or interactions close to chat, are:
- SMS or Textmessaging
- Microsoft Teams
- Slack
- Lync for Business (formerly Skype)
- Facebook Chat
- WhatsApp
- Snapchat 
- Tinder
- Happn
- Chat Support on company web pages
I would also claim that I am in no way unique using many different chat apps throughout the day, and would also therefore claim that chat has become a natural way of interacting with other people.

I would also claim that Chat has a low threshold, lower than using the phone, and it is more instant than using mail. As mail, is in a lot of ways an asynchronous way of communicating.

Speech Recognition:
I rarely use speech recognition, in other ways than as a "party trick". To me this is partly because of the social stigma talking to no one out and about, but the other important part is the lacking precision when I have tried it.

In this project I have used the examples from Google, for their Google Speech API
(REFERENCE)

I have altered the code of the Recognize listener to post the results to a Web API running locally.

But the actual recognized words, has been very limited, typically it has been limited to one, or two words, and not whole sentences.
I have tried both using the laptops integrated microphone, but also a couple of different headsets. None of them have yielded very impressive results.
Caveat: It could be that the actual implemtation example from Google is not the best one there is for my use case, and that improving on the recording part of it, and streaming data would be better. However, spending time working on this has not been a priority, so the actual implementation does nothing more than try to recognize a movie and then post the results about it to a channel on Slack.




%----------------------------------------------------------------------------------------
%	CHAPTER 4 Interactive System: Development
%----------------------------------------------------------------------------------------

\chapterimage{head3.png}
\chapter{Interactive System: Development}
I chose to start with an implementation of Speech Recognition, as this was the initial idea of the project.
Some quick googling, and I landed on Googles Speech API (REFERENCE). There is an online tester (LINK)

First I had to enable the Speech API in the Google Cloud console. The API is limited to a certain amount of seconds in the free tier.

First I tried using the Node JS library, but I was not able to get the sound recorder to work properly on my Windows 10 laptop. (It might work better on a linux or Mac, but I am not sure of that)

I then moved on to C\# as I am somewhat familiar with that language. The speech recorder part of that worked much better, and I was able to get the example code to recognize my speech quite quickly.
After I got it working, I modified the code to post the recognized speech to the designated Slack channel.

I selected Slack, as I have developed a couple of things for posting to its interface earlier, and the familiarty meant I was able to test and implement the speech recognition quicker, than had I selected Facebook Messenger API, or Microsofts similar products.
The selection here was then based on familiarity and not based on necessarily "Best of breed" for my specific task.

Next part of the development was the "omdbsearcher". Here I selected OMDB (Open Movie Database) and had to donate a couple of dollars to get access to their API. There are implementations for IMDB, but they are not, to my knowledge authorized by IMDB, and I was not interested in spending time on a non official api.

IMDB Also has some of its data dumped to a FTP on regular intervals, so creating an own database could have been a future work part for this project.

The omdbsearcher I implemented in Node JS, as this is the language I can program in fastest. I tested a couple of request with Postman (reference) which is a tool to create web requests to different endpoints. Postman also has a nice feature where you can get the code for replicating the request in different languages.

I then created a local webserver serving the incoming recognized speech, and then trying to search for them on OMDB. If it got a hit, it posted it to the Slack Channel I had selected.

Description of how it was developed
o	If coding project, then include key code snippets and discuss them
o	If using a prototype tool, step-by-step description of how the interactive system was created



%----------------------------------------------------------------------------------------
%	CHAPTER 5 Interactive System: Evaluation
%----------------------------------------------------------------------------------------

\chapterimage{head3.png}
\chapter{Interactive System: Evaluation}
•	Evaluation plan for the interactive system incorporating at least 5 design principles from the course
Unfortunately, even though the TextBlob (reference) python Library is good, it is still lacking with regards to the goal of my project.
It is able to, sometimes recognize the noun phrase and sentiment, but it is not always the case.
Also keeping context of a ongoing discussion is not something I have looked into at all.
However, I do not think this will not be possible to solve in the future, however they might be "hard" problems.

The speech recognition part of the project, albeit fun, did not fulfill my hopes for what it was able to do. For me to be able to do that, I would probably have to work some time on the actual recording of the audio, and how to optimize this part. Also having the audio recorder being able to switch on and off based on speech patterns.
Also recognizing who actually speaks, is not something I have seen covered in the documentation.
For recognizing users, a chatroom with identified users, is easier.

%----------------------------------------------------------------------------------------
%	CHAPTER 6 Ten Meaningful Screens
%----------------------------------------------------------------------------------------

\chapterimage{head3.png}
\chapter{Ten Meaningful Screens}
List 10 meaningful screens that your interactive system provides and specify the operational sequence to find them.

%----------------------------------------------------------------------------------------
%	CHAPTER 7 Five Actionable Events
%----------------------------------------------------------------------------------------

\chapterimage{head3.png}
\chapter{Five Actionable Events}
List and discuss 5 actionable events encountered in your interactive system.

%----------------------------------------------------------------------------------------
%	CHAPTER 8 Two Valuable Outcomes
%----------------------------------------------------------------------------------------

\chapterimage{head3.png}
\chapter{Two Valuable Outcomes}
List and discuss two valuable organizational outcomes from using your interactive system: one in the short-term and the other in the long-term. 

Generally speaking I would say that, unfortunately my interactive system does not provide that valuable of outcomes.
However here I will try to describe the two, that I thought came the closest.

1:
Speech recognition of movies. Being able to search for, and find ratings for movies without having to use the telephone, could be something I might use.

Especially if there was a way of starting to play the movie automatically, once I was happy with the selection.

However for a home theater version of this to work, the description and possibly the ability to play a trailer would have to be done on the Television, and maybe not in a Slack channel.

2:
Being able to detect how people talk about movies, and then recommending them movies based on what they might also like i think could be very valuable.

The difficulty of course would be the level of Natural Language Processing to make this as seamless as I would want it to be.

My rather simple implementation of using a Python library to detect noun phrases, and sentiment values, can work on some of the sentences people can say, and write about movies, but it does not keep context.

For both of these outcomes I would, however say, that for me it has been valuable to learn, and try to make the mostly "off-the-shelf" technology work together to create these solutions. And with the amount of time available I am rather happy with what I was able to create.

Future work in this might be interesting, and I do believe that better time, and better knowledge of the API's libraries etcetera would mean a better performing Interactive System.


\vfill
\end{document}