%%%%%%%%%%%%%%%%%%%%%%%%%%%%%%%%%%%%%%%%%
%  My documentation report
%  Objetive: Explain what I did and how, so someone can continue with the investigation
%
% Important note:
% Chapter heading images should have a 2:1 width:height ratio,
% e.g. 920px width and 460px height.
%
%%%%%%%%%%%%%%%%%%%%%%%%%%%%%%%%%%%%%%%%%

%----------------------------------------------------------------------------------------
%	PACKAGES AND OTHER DOCUMENT CONFIGURATIONS
%----------------------------------------------------------------------------------------

\documentclass[11pt,fleqn]{book} % Default font size and left-justified equations

\usepackage[top=3cm,bottom=3cm,left=3.2cm,right=3.2cm,headsep=10pt,letterpaper]{geometry} % Page margins

\usepackage{xcolor} % Required for specifying colors by name
\definecolor{ocre}{RGB}{52,177,201} % Define the orange color used for highlighting throughout the book

% Font Settings
\usepackage{avant} % Use the Avantgarde font for headings
%\usepackage{times} % Use the Times font for headings
\usepackage{mathptmx} % Use the Adobe Times Roman as the default text font together with math symbols from the Sym­bol, Chancery and Com­puter Modern fonts

\usepackage{microtype} % Slightly tweak font spacing for aesthetics
\usepackage[utf8]{inputenc} % Required for including letters with accents
\usepackage[T1]{fontenc} % Use 8-bit encoding that has 256 glyphs

% Bibliography
\usepackage[style=alphabetic,sorting=nyt,sortcites=true,autopunct=true,babel=hyphen,hyperref=true,abbreviate=false,backref=true,backend=biber]{biblatex}
\addbibresource{bibliography.bib} % BibTeX bibliography file
\defbibheading{bibempty}{}

\input{structure} % Insert the commands.tex file which contains the majority of the structure behind the template

\begin{document}

%----------------------------------------------------------------------------------------
%	TITLE PAGE
%----------------------------------------------------------------------------------------

\begingroup
\thispagestyle{empty}
\AddToShipoutPicture*{\put(0,0){\includegraphics[scale=1.25]{frontpage3}}} % Image background
\centering
\vspace*{5cm}
\par\normalfont\fontsize{32}{32}\sffamily\selectfont
\textbf{Filmchat}\\
{\LARGE Movie recommendation as NUI}\par % Book title
\vspace*{1cm}
{\Huge Tarjei Utnes}\par % Author name
\endgroup

%----------------------------------------------------------------------------------------
%	TABLE OF CONTENTS
%----------------------------------------------------------------------------------------

\chapterimage{head3.png} % Table of contents heading image

\pagestyle{empty} % No headers

\tableofcontents % Print the table of contents itself

%\cleardoublepage % Forces the first chapter to start on an odd page so it's on the right

\pagestyle{fancy} % Print headers again

%----------------------------------------------------------------------------------------
%	CHAPTER 1 Project Description
%----------------------------------------------------------------------------------------

\chapterimage{head3.png} % Chapter heading image

\chapter{Project Description}
This project is aimed for use at home, or possibly on-line in chat rooms or forums. The goal of the project is to interpret and come up with suggestions for which movies, or TV series its users should watch. I have limited the scope in this project to implement parts of the project, because of the available time given.
There are competing or projects existing that are close to doing what this project is aiming to do. Below is a non-exhaustive list of such projects, and solutions.
\section{“And Chill” Netflix recommendation bot}\index{AndChill}
The “And Chill” chatbot:
\begin{quote}
Well, that’s when the Facebook chatbot And Chill comes in handy. The bot asks you to suggest a film you like with a few reasons why, or to simply explain what you’re looking for in your next movie.  
Within a few moments, the bot should issue you with a short selection of trailers. It sometimes takes a while to compute, but it’s still a pretty efficient way of narrowing down Netflix’s massive library.
That said, if it’s running at peak capacity, you’ll have to wait your turn before it takes your requests. A representative for the app told a Facebook commenter last month that they were working on scaling capacity. (Huffington Post 2016)
\end{quote}
\section{Filmgrail}\index{Filmgrail}
Filmgrail is an app to check where you are able to watch the movie or television series you have selected. What makes Filmgrail unique in a Norwegian context, is that it covers many of the Norwegian streaming services. (Aftenposten, Filmgrail.com). At the moment, however 
\section{Google Assistant}\index{googleassistant}
Google Assistant can also recommend movies.



%----------------------------------------------------------------------------------------
%	CHAPTER 2 Interactive System Description
%----------------------------------------------------------------------------------------
\chapterimage{head3.png}

\chapter{Interactive System: Description}
There are two parts implemented in this system
The first part listens to a given channel on Slack, using the outgoing webhook (REFERENCE). 
The second part uses Googles Speech API (reference) and listens to the microphone. It then tries to search for the terms in the recognized speech and if it finds a movie within it it posts it to the same Slack channel.

The though of using both of these ways of integrating, was to be able to understand natural language surrounding how people talk and chat about movies, and then later on have the different parts use the same components.



%----------------------------------------------------------------------------------------
%	CHAPTER 3 Interactive System Design
%----------------------------------------------------------------------------------------

\chapterimage{head3.png}
\chapter{Interactive System: Design}
•	Sketches
•	Design rationale for the interface
•	Cost/benefit discussion of selected interaction techniques

Cost / Benefit of selected interaction techniques
Chat:
Chat is in my view has become a natural way of interacting, several times a day, I interact through chat, be it privately and also work related. A non-exhaustive list of some of the chat interaction I use during a day, or interactions close to chat, are:
- SMS or Textmessaging
- Microsoft Teams
- Slack
- Lync for Business (formerly Skype)
- Facebook Chat
- WhatsApp
- Snapchat 
- Tinder
- Happn
- Chat Support on company web pages
I would also claim that I am in no way unique using many different chat apps throughout the day, and would also therefore claim that chat has become a natural way of interacting with other people.

I would also claim that Chat has a low threshold, lower than using the phone, and it is more instant than using mail. As mail, is in a lot of ways an asynchronous way of communicating.

Speech Recognition:
I rarely use speech recognition, in other ways than as a "party trick". To me this is partly because of the social stigma talking to no one out and about, but the other important part is the lacking precision when I have tried it.

I would use the Speech Recognition in much the same way as I would use the 

%----------------------------------------------------------------------------------------
%	CHAPTER 4 Interactive System: Development
%----------------------------------------------------------------------------------------

\chapterimage{head3.png}
\chapter{Interactive System: Development}
Description of how it was developed
o	If coding project, then include key code snippets and discuss them
o	If using a prototype tool, step-by-step description of how the interactive system was created


%----------------------------------------------------------------------------------------
%	CHAPTER 5 Interactive System: Evaluation
%----------------------------------------------------------------------------------------

\chapterimage{head3.png}
\chapter{Interactive System: Evaluation}
•	Evaluation plan for the interactive system incorporating at least 5 design principles from the course

%----------------------------------------------------------------------------------------
%	CHAPTER 6 Ten Meaningful Screens
%----------------------------------------------------------------------------------------

\chapterimage{head3.png}
\chapter{Ten Meaningful Screens}
List 10 meaningful screens that your interactive system provides and specify the operational sequence to find them.

%----------------------------------------------------------------------------------------
%	CHAPTER 7 Five Actionable Events
%----------------------------------------------------------------------------------------

\chapterimage{head3.png}
\chapter{Five Actionable Events}
List and discuss 5 actionable events encountered in your interactive system.

%----------------------------------------------------------------------------------------
%	CHAPTER 8 Two Valuable Outcomes
%----------------------------------------------------------------------------------------

\chapterimage{head3.png}
\chapter{Two Valuable Outcomes}
List and discuss two valuable organizational outcomes from using your interactive system: one in the short-term and the other in the long-term. 




\vfill
%\textit{Wish you all the best, Andrea Hidalgo}
\end{document}